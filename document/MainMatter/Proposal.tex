\chapter{Propuesta}\label{chapter:proposal}

El método empleado consiste en dos fases. Una primera fase en la que se buscan los modelos que
maximizan una métrica de diversidad mientras que se optimiza la precisión de los mismos. Luego una
segunda fase utiliza los \emph{n} modelos anteriormente obtenidos para ser ensamblados utilizando un
enfoque multiobjetivo donde se optimizan simultaneamente una metrica de equidad y la precision.

\section{Generación de modelos con diversidad alta para una tarea determinada}

Para esta fase utilizamos una tecnica que puede pensarse como una generalización basada en AutoML de
la idea presentada en \cite{SnapshotEnsembles}. En particular se utiliza una modificacion de
Probabilistic Gramatical Evolution Search que permite controlar características de la población como
por ejemplo en este caso mantener en todo momento las n soluciones mas diversas que se han explorado
durante el proceso de busqueda. Esto se logra manteniendo en cada generacion del algoritmo los n mejores
modelos en cuanto a una funcion de ranking.

La función de ranking que se utiliza para ordernar las soluciones de acorde a su diversidad. En este
momento utiliza un enfoque greedy, en el cual se comienza ordenando las soluciones según la precisión
en un conjunto de evaluación, y luego se procede a asignarle un rank a cada clasificador en funcion
de la diversidad que presenta este con el conjunto de soluciones seleccionadas hasta el momento,
priorizando a los clasificadores con mayor precisión y añadiendo al final de este proceso el
clasificador mejor rankeado al conjunto de soluciones seleccionadas, este proceso se repite hasta
que todos los clasificadores son asignados un rank.

Para computar la diversidad relativa entre un par de clasificadores se utiliza una de las metricas
de diversidad descritas en \cite{DiversityMeassures}, ajustada de forma que refleje la diversidad
relativa entre pares de clasificadores en vez de la diversidad de todo el conjunto.

\section{Ensamblado de los modelos utilizando multiples criterios}

Para la construcción del ensemble a partir de los modelos obtenidos en la primera fase se utiliza un
hibrido entre Probabilistic Gramatical Evolution y NSGA-II. En particular tomamos los metodos de
Non-dominated Sorting y Crowding Distance de NSGA-II y los incorporamos a Probabilistic Gramatical
Evolution, en particular al paso de decidir los individuos que pasan a la siguiente generacion.

Non-dominated Sorting consiste en ordenar los individuos de forma tal que sean priorizados aquellos
individuos que sean dominados por la menor cantidad de individuos de la población. Donde un individuo
domina a otro si este mejora al menos una de las metricas a optimizar, sin comprometer ninguna de las
otras.
