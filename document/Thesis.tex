\documentclass[12pt,oneside]{uhthesis}
\usepackage{subfigure}
\usepackage[ruled,lined,linesnumbered,titlenumbered,algochapter,spanish,onelanguage]{algorithm2e}
\usepackage{amsmath}
\usepackage{amssymb}
\usepackage{amsbsy}
\usepackage{caption,booktabs}
\captionsetup{ justification = centering }
%\usepackage{mathpazo}
\usepackage{float}
\setlength{\marginparwidth}{2cm}
\usepackage{todonotes}
\usepackage{listings}
\usepackage{xcolor}
\usepackage{multicol}
\usepackage{graphicx}
\floatstyle{plaintop}
\restylefloat{table}
\addbibresource{Bibliography.bib}
% \setlength{\parskip}{\baselineskip}%
\renewcommand{\tablename}{Tabla}
\renewcommand{\listalgorithmcfname}{Índice de Algoritmos}
%\dontprintsemicolon
\SetAlgoNoEnd

\definecolor{codegreen}{rgb}{0,0.6,0}
\definecolor{codegray}{rgb}{0.5,0.5,0.5}
\definecolor{codepurple}{rgb}{0.58,0,0.82}
\definecolor{backcolour}{rgb}{0.95,0.95,0.92}

\lstdefinestyle{mystyle}{
    backgroundcolor=\color{backcolour},   
    commentstyle=\color{codegreen},
    keywordstyle=\color{purple},
    numberstyle=\tiny\color{codegray},
    stringstyle=\color{codepurple},
    basicstyle=\ttfamily\footnotesize,
    breakatwhitespace=false,         
    breaklines=true,                 
    captionpos=b,                    
    keepspaces=true,                 
    numbers=left,                    
    numbersep=5pt,                  
    showspaces=false,                
    showstringspaces=false,
    showtabs=false,                  
    tabsize=4
}

\lstset{style=mystyle}

\title{Mitigación de sesgos con ensembles y optimización multiobjetivo}
\author{\\\vspace{0.25cm}Jorge Mederos Alvarado}
\advisor{\\\vspace{0.25cm}Juan Pablo Consuegra Ayala\\\vspace{0.2cm}Alejandro Piad Morfis}
\degree{Licenciado en (Matemática o Ciencia de la Computación)}
\faculty{Facultad de Matemática y Computación}
\date{2022-06-30\\\vspace{0.25cm}\href{https://github.com/jmederosalvarado/thesis}{github.com/jmederosalvarado/thesis}}
\logo{Graphics/uhlogo}
\makenomenclature

\renewcommand{\vec}[1]{\boldsymbol{#1}}
\newcommand{\diff}[1]{\ensuremath{\mathrm{d}#1}}
\newcommand{\me}[1]{\mathrm{e}^{#1}}
\newcommand{\pf}{\mathfrak{p}}
\newcommand{\qf}{\mathfrak{q}}
%\newcommand{\kf}{\mathfrak{k}}
\newcommand{\kt}{\mathtt{k}}
\newcommand{\mf}{\mathfrak{m}}
\newcommand{\hf}{\mathfrak{h}}
\newcommand{\fac}{\mathrm{fac}}
\newcommand{\maxx}[1]{\max\left\{ #1 \right\} }
\newcommand{\minn}[1]{\min\left\{ #1 \right\} }
\newcommand{\lldpcf}{1.25}
\newcommand{\nnorm}[1]{\left\lvert #1 \right\rvert }
\renewcommand{\lstlistingname}{Ejemplo de código}
\renewcommand{\lstlistlistingname}{Ejemplos de código}

\begin{document}

\frontmatter
\maketitle

% \include{FrontMatter/Dedication}
% \include{FrontMatter/Thanks}
% \include{FrontMatter/SupervisorOpinion}
\include{FrontMatter/Abstract}
\include{FrontMatter/Contents}

\mainmatter

\chapter*{Introducción}\label{chapter:introduction}
\addcontentsline{toc}{chapter}{Introducción}

El aumento de la importancia y el área de uso de los algoritmos de aprendizaje automático ha alcanzado un nivel en el que estos forman parte de la vida diaria de las personas (a veces sin que las mismas lo perciban). Esto ha influido en el interés por parte de la comunidad científica en estudiar a fondo dichos algoritmos, sus ventajas y limitaciones. La popularidad y la influencia en la vida diaria humana han llevado a varias preocupaciones ¿Hasta que punto podemos confiar en las decisiones tomadas por una máquina entrenada por un algoritmo de aprendizaje automático? ¿Cómo podemos saber que las decisiones que toma esta máquina son justas? Se puede entender como justicia o equidad a la ausencia de favoritismo o prejuicios hacia un individuo o grupo, basado en características inherentes o adquiridas.\\
Los algoritmos usan conjuntos de datos sacados de las sociedades humanas, por tanto, los sesgos existentes en estas pueden potencialmente verse reflejados en las decisiones tomadas por la máquina entrenada. Estos sesgos pueden, incluso, verse amplificados.\\
Lo anterior deja ver la importancia que tiene para la sociedad el estudio de las formas de eliminación de los sesgos, ya sean, entre muchos otros, raciales o de género. La responsabilidad de esto cae en manos de los desarrolladores y el presente trabajo pretende dar ciertos aportes a la solución de dicho problema. Un ejemplo claro del riesgo de ignorar lo anterior es la popularización de representaciones semánticas pre-entrenadas en registros históricos, contenedores de sesgos sociales.\\
En los últimos años ha crecido el interés por el desarrollo de técnicas que permitan identificar y eliminar los sesgos en algoritmos de aprendizaje automático, pero que mantengan al mismo tiempo la precisión de los modelos.\\

Una de las principales desventajas del Aprendizaje de Máquina es el nivel de preparación que tiene que
tener un investigador o ingeniero para sacar provecho de las técnicas en problemas reales.
Para ello se han venido explorando muy recientemente alternativas que permitan
automatizar el proceso de selección y entrenamiento de modelos de aprendizaje de máquina,
permitiendo a los ingenieros e investigadores enfocarse en los detalles que no pueden ser
automatizados o son de dominio muy específico y así atacar los problemas de una forma más
ágil. A este tipo de técnicas se les conoce como técnicas de AutoML y están jugando un
papel fundamental en la democratización de la inteligencia artificial y el aprendizaje de
máquinas.\\
Como parte de este trabajo se pretende utilizar una biblioteca de AutoML llamada AutoGoal, para asistir en el objetivo fundamental de
obtener un método basado en ensembles que permita construir modelos con alta precisión
para problemas de clasificación. Poniendo énfasis en la explotación de la diversidad entre los
integrantes de los ensembles y permitiendo la optimización de mas de una métrica, abriendo
paso a aplicaciones como la mitigación de los sesgos antes mencionados.\\
Para lograr el objetivo planteado, luego de una amplia curva de investigación y aprendizaje, se llegó a una solución que hace uso de algoritmos como Non-dominated Sorting y Probabilistic Grammatical Evolution.

\section*{Motivación}

Un modelo de aprendizaje de maquina se entrena con el objetivo de optimizar una unica metrica, en la mayoria de los casos la precision. Esto significa que los modelos aprenden muy bien los patrones que se presentan en los datos de entrenamiento, incluyendo aquellos patrones que representan sesgos y prejuicios que estan desafortunadamente presente en la sociedad y por ende en los datos recopilados, en algunos casos incluso amplifican estos patrones negativos. Son varias la tecnicas que se han explorado para resolver este problema, algunas se enfocan en un preprocesamiento de los datos para eliminar aquellos elementos que puedan inducir un sesgo en el modelo, otras realizan variacinoes en el metodo de entrenamiento con el mismo objetivo. Sin embargo permanece relativamente poco explorado el uso de tecnicas de optimizacion multiobjetivo que permitan al modelo optimizar hasta encontrar un buen balance entre cuan justo es y cuan preciso.

Otra tecnica que ha demostrado ser de gran utilidad en la prevencion de los sesgos en los modelos de aprendizaje de maquina es la construccion de ensamblados de multiples modelos que maximizan la varianza entre si, por lo que se minimiza el sesgo del ensamblado final.

\section*{Problematica}

A pesar de que existe AutoGOAL, una biblioteca de AutoML, que permite obtener modelos para resolver problemas arbitrarios utilizando entre otras tecnicas aprendizaje de maquina. No existe una biblioteca o herramienta que permita resolver de principio a fin un problema de clasificacion utilizando aprendizaje de maquina y donde exita alguna garantia de que el modelo aprendido sea justo.

\section*{Objetivo general}

Proponer una herramienta que permita resolver problemas de clasificacion utilizando aprendizaje de maquina y que permita garantizar que el modelo aprendido sea justo.

\section*{Objetivos especifico}

\begin{itemize}
    \item Encontrar modelos que maximicen la varianza para minimizar el sesgo.
    \item Metodos basados en metaheuristicas para optimizar los modelos utilizando simultaneamente metricas de equidad y precision.
    \item Explorar adicion de optimizacion multiobjetivo a AutoGOAL para que el modelo aprendido sea justo.
    \item Metodos basados en la combinacion de diferentes metricas en una sola, para poder aprovechar los multiples metodos de optimizacion que existen.
\end{itemize}

\include{MainMatter/Background}
\chapter{Propuesta}\label{chapter:proposal}

El método empleado consiste en dos fases. Una primera fase en la que se buscan los modelos que
maximizan una métrica de diversidad mientras que se optimiza la precisión de los mismos. Luego una
segunda fase utiliza los \emph{n} modelos anteriormente obtenidos para ser ensamblados utilizando un
enfoque multiobjetivo donde se optimizan simultaneamente una metrica de equidad y la precision.

\section{Generación de modelos con diversidad alta para una tarea determinada}

Para esta fase utilizamos una tecnica que puede pensarse como una generalización basada en AutoML de
la idea presentada en \cite{SnapshotEnsembles}. En particular se utiliza una modificacion de
Probabilistic Gramatical Evolution Search que permite controlar características de la población como
por ejemplo en este caso mantener en todo momento las n soluciones mas diversas que se han explorado
durante el proceso de busqueda. Esto se logra manteniendo en cada generacion del algoritmo los n mejores
modelos en cuanto a una funcion de ranking.

La función de ranking que se utiliza para ordernar las soluciones de acorde a su diversidad. En este
momento utiliza un enfoque greedy, en el cual se comienza ordenando las soluciones según la precisión
en un conjunto de evaluación, y luego se procede a asignarle un rank a cada clasificador en funcion
de la diversidad que presenta este con el conjunto de soluciones seleccionadas hasta el momento,
priorizando a los clasificadores con mayor precisión y añadiendo al final de este proceso el
clasificador mejor rankeado al conjunto de soluciones seleccionadas, este proceso se repite hasta
que todos los clasificadores son asignados un rank.

Para computar la diversidad relativa entre un par de clasificadores se utiliza una de las metricas
de diversidad descritas en \cite{DiversityMeassures}, ajustada de forma que refleje la diversidad
relativa entre pares de clasificadores en vez de la diversidad de todo el conjunto.

\section{Ensamblado de los modelos utilizando multiples criterios}

Para la construcción del ensemble a partir de los modelos obtenidos en la primera fase se utiliza un
hibrido entre Probabilistic Gramatical Evolution y NSGA-II. En particular tomamos los metodos de
Non-dominated Sorting y Crowding Distance de NSGA-II y los incorporamos a Probabilistic Gramatical
Evolution, en particular al paso de decidir los individuos que pasan a la siguiente generacion.

Non-dominated Sorting consiste en ordenar los individuos de forma tal que sean priorizados aquellos
individuos que sean dominados por la menor cantidad de individuos de la población. Donde un individuo
domina a otro si este mejora al menos una de las metricas a optimizar, sin comprometer ninguna de las
otras.

\include{MainMatter/Implementation}

\backmatter

\include{BackMatter/Conclusions}
\include{BackMatter/Recomendations}
\include{BackMatter/Bibliography}

\end{document}
